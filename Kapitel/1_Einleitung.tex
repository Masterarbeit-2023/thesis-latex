\chapter{Einleitung}

\section{Hintergrund und Motivation}
Selbst in einer schnelllebigen technologischen Landschaft bleibt eine Konstante bestehen: 
die kontinuierliche Evolution von Webanwendungen und ihrer zugrunde liegenden Infrastrukturen. 
Mit der stetigen Weiterentwicklung von Softwareentwicklungsmethoden, Bereitstellungstechnologien 
und Cloud-Plattformen haben sich auch die Ansätze zur Entwicklung und Bereitstellung von Webanwendungen stark verändert. 
In diesem Kontext gewinnt das Konzept der infrastruktur-agnostischen Entwicklung und Bereitstellung zunehmend an Bedeutung. 
Infrastruktur-agnostische Ansätze ermöglichen es, Webanwendungen unabhängig von den spezifischen technologischen Details der zugrunde liegenden Infrastruktur zu gestalten und bereitzustellen. 
Dies eröffnet Chancen für eine erhöhte Flexibilität, Skalierbarkeit und Effizienz in der Softwareentwicklung und -bereitstellung.


\section{Zielsetzung der Arbeit}
In dieser Masterarbeit wird die Thematik der infrastruktur-agnostischen Entwicklung und Bereitstellung von Webanwendungen umfassend untersucht. 
Dabei werden die zugrunde liegenden Konzepte, Methoden und Technologien analysiert, die es ermöglichen, Webanwendungen von den Details der Infrastruktur zu abstrahieren. 
Ein besonderer Fokus liegt dabei auf der Identifizierung von Best Practices, Herausforderungen und potenziellen Lösungsansätzen im Zusammenhang mit der Umsetzung infrastruktur-agnostischer Ansätze. 
Des Weiteren werden Fallstudien und Praxisbeispiele aus verschiedenen Industriezweigen betrachtet, um ein umfassendes Verständnis für die Anwendung und Auswirkungen dieser Ansätze zu entwickeln.

Indem diese Arbeit die Vor- und Nachteile, die möglichen Auswirkungen auf die Softwareentwicklung und die langfristigen Perspektiven der infrastruktur-agnostischen Entwicklung und Bereitstellung von Webanwendungen beleuchtet, trägt sie zur Erweiterung des Wissens in diesem sich rasch entwickelnden Bereich bei. 
Letztendlich wird diese Arbeit dazu beitragen, Entwicklern, Unternehmen und Entscheidungsträgern dabei zu helfen, fundierte Entscheidungen im Hinblick auf die Wahl geeigneter Ansätze für die Entwicklung und Bereitstellung ihrer Webanwendungen zu treffen, während sie von den Vorteilen einer infrastruktur-agnostischen Herangehensweise profitieren.
