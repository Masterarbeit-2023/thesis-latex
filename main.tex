%=======================================================================

% Template for Thesis at Computer Sciencte at
% Ostfalia University of Applied Science
% Edit: Dirk J. Lehmann, Kai Michael Blum

%=======================================================================
\documentclass[
		11pt,
		a4paper, 
		oneside, % twoside use two sided for book layout
		% openany,
		toc=listofnumbered,
		toc=bibliography, 
		toftofoc,
		headings=small,
		headings=twolineappendix
]{scrbook}
\usepackage[utf8]{inputenc}

%=======================================================================

\newif\ifdraft
\draftfalse % Sagt aus, dass dieses Dokument ein Entwurf ist. Somit wird todonotes aktiviert. Zum deaktivieren diese Zeile auskommentieren oder auf \draftfalse setzen.

%=======================================================================

\input{Meta/praeampel}

%=======================================================================

\newcommand{\documenttitle}{Infrastruktur-agnostische Entwicklung und Bereitstellung von Webanwendung}
\newcommand{\documentname}{Niklas Röske}
\newcommand{\documentmanr}{70456600}
\newcommand{\documentmodul}{}
\newcommand{\documentstudiengang}{Informatik}
\newcommand{\documentachievement}{zur Erlangung des akademischen Grades:\\Master of Science\\}




\newcommand{\documentbetreuer}{Prof. Dr. Hans Grönniger}
\newcommand{\documentbetreuerzwei}{Prof. Dr. Bernd Müller}
\newcommand{\documentsemester}{2023}

\newcounter{documentfakultaet}
\setcounter{documentfakultaet}{1} % 0 = Bau Wasser Boden, 1 = Informatik

%=======================================================================


\begin{document}	

    % = = = = =	= = = = = = = = = = = = = = = = = = = = = = = =
	
	\input{Kapitel/0_0_Deckblatt}
    
    % = = = = =	= = = = = = = = = = = = = = = = = = = = = = = =	
	
	\frontmatter
	\pagenumbering{Roman}
	\ifnum\value{documentfakultaet}=1
		\input{Kapitel/0_erklaerung}
		\markboth{Erklärung}{}
	\fi
    
    % = = = = =	= = = = = = = = = = = = = = = = = = = = = = = =
	
	\input{Kapitel/0_Abstract}
    
    % = = = = =	= = = = = = = = = = = = = = = = = = = = = = = =	
	
	% disable header
	\fancyhead[L]{\chaptermark}  % L -> Left part
	\fancyhead[RO]{\chaptermark} % RO -> Right part on Odd pages
	
	\tableofcontents % Inhaltsverzeichnis
	
	% = = = = =	= = = = = = = = = = = = = = = = = = = = = = = =
	
	\input{Kapitel/0_Abkuerzungsverzeichnis}
	
	% = = = = =	= = = = = = = = = = = = = = = = = = = = = = = =
	
	\mainmatter
	
	% reanable header
	\fancyhf{}
	\rhead{\nouppercase\rightmark}
	\lhead{\nouppercase\leftmark}
	
	\fancypagestyle{plain}{}
	%\fancyhead[LE]{\chaptermark} % Chapter in header Left
	%\fancyhead[RE]{\chaptermark} % Page number in header Right
	%\fancyfoot[LE]{\thepage}     % LE -> Left part on Even pages
	\fancyfoot[RO]{\thepage}     % RO -> Right part on Odd pages
	
	% = = = = =	= = = = = = = = = = = = = = = = = = = = = = = =
	
	% reset list of used acronyms
	\acresetall
	
	% = = = = =	= = = = = = = = = = = = = = = = = = = = = = = =
	
	% Kapitel einfügen
	\chapter{Einleitung}

\section{Hintergrund und Motivation}
Selbst in einer schnelllebigen technologischen Landschaft bleibt eine Konstante bestehen: 
die kontinuierliche Evolution von Webanwendungen und ihrer zugrunde liegenden Infrastrukturen. 
Mit der stetigen Weiterentwicklung von Softwareentwicklungsmethoden, Bereitstellungstechnologien 
und Cloud-Plattformen haben sich auch die Ansätze zur Entwicklung und Bereitstellung von Webanwendungen stark verändert. 
In diesem Kontext gewinnt das Konzept der infrastruktur-agnostischen Entwicklung und Bereitstellung zunehmend an Bedeutung. 
Infrastruktur-agnostische Ansätze ermöglichen es, Webanwendungen unabhängig von den spezifischen technologischen Details der zugrunde liegenden Infrastruktur zu gestalten und bereitzustellen. 
Dies eröffnet Chancen für eine erhöhte Flexibilität, Skalierbarkeit und Effizienz in der Softwareentwicklung und -bereitstellung.


\section{Zielsetzung der Arbeit}
In dieser Masterarbeit wird die Thematik der infrastruktur-agnostischen Entwicklung und Bereitstellung von Webanwendungen umfassend untersucht. 
Dabei werden die zugrunde liegenden Konzepte, Methoden und Technologien analysiert, die es ermöglichen, Webanwendungen von den Details der Infrastruktur zu abstrahieren. 
Ein besonderer Fokus liegt dabei auf der Identifizierung von Best Practices, Herausforderungen und potenziellen Lösungsansätzen im Zusammenhang mit der Umsetzung infrastruktur-agnostischer Ansätze. 
Des Weiteren werden Fallstudien und Praxisbeispiele aus verschiedenen Industriezweigen betrachtet, um ein umfassendes Verständnis für die Anwendung und Auswirkungen dieser Ansätze zu entwickeln.

Indem diese Arbeit die Vor- und Nachteile, die möglichen Auswirkungen auf die Softwareentwicklung und die langfristigen Perspektiven der infrastruktur-agnostischen Entwicklung und Bereitstellung von Webanwendungen beleuchtet, trägt sie zur Erweiterung des Wissens in diesem sich rasch entwickelnden Bereich bei. 
Letztendlich wird diese Arbeit dazu beitragen, Entwicklern, Unternehmen und Entscheidungsträgern dabei zu helfen, fundierte Entscheidungen im Hinblick auf die Wahl geeigneter Ansätze für die Entwicklung und Bereitstellung ihrer Webanwendungen zu treffen, während sie von den Vorteilen einer infrastruktur-agnostischen Herangehensweise profitieren.

	% ----------------------------------------------------------
% ----------------------------------------------------------
\chapter{Theoretischer Hintergrund}

\section{Web-Anwendungen}
\label{sec:visual_analytics}

\section{Infrastruktur unabhängige Entwicklung}

\section{Herausforderungen bei der Entwicklung}

\cite{Sammon.1969}\\
\cite{Demartines.1997}\\
	\chapter{Methodik}

\section{Auswahl Entwicklungstechnologien}

\section{Architektur und Design der Web-Anwendung}

\section{Implementierung un Testing?}
	\chapter{Empfehlungen und bewährte Praktiken}
\label{sec:approach}

\section{Best Practices}

\section{Technische Aspekte}
	\chapter{Fallstudie}


\section{Beschreibung der Anwendung}

\section{Bewertung der Ergebnisse}

	\chapter{Diskussion}

\section{Wichtigste Erkenntnisse}

\section{Beantwortung der Forschungsfrage}

\section{Kritische Bewertung der Empfehlung}
	\chapter{Ausblick}

Your discussion goes here ...

    % = = = = =	= = = = = = = = = = = = = = = = = = = = = = = =

    % Bilbliography
	\bibliographystyle{dinat} 
	\bibliography{literatur}
	
	% = = = = =	= = = = = = = = = = = = = = = = = = = = = = = =
	
	\ifnum\value{documentfakultaet}=0
		\input{erklaerung}
		\markboth{Erklärung}{}
	\fi
	
	% = = = = =	= = = = = = = = = = = = = = = = = = = = = = = =
	
    %\addtocontents{toc}{\cftpagenumbersoff{chapter}} % Keine Seitenzahl im Inhaltsverzeichnis
	\begin{appendices}              % Anhang Titel-Seite
	\end{appendices}
	
	\fancyfoot[R] {\thepage}        % Seitenzahl Right
	
	\appendix
	\input{Kapitel/100_Anhang}
	%\postappendix
	
	% = = = = =	= = = = = = = = = = = = = = = = = = = = = = = =
	
\end{document}